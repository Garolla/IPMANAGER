%%%%%%%%%%%%%%%%%%%%%%%%%%%%%%%%%%%%%%%%%%%%%%%%%%%%
% This will help you in writing your homebook
% Remember that the character % is a comment in latex
%
% chapter 1

\label{chap1}
\chapter{Abstract}

%%%%%%%%%%%%%%%%%%%%%%%%%%%%%%%%%%%%%%%%%%%%%%%%%%%%%%%%%%%
% you can organize a chapter using sections -> \section{Simulating an inverter}
% or subsections -> \subsection{simulating a particular type of inverter}

%%%%%%   First section

%\section{Milestones}

\begin{table}[h]
	\begin{tabular}{p{2.1cm}|  p{13.5cm}}
\multicolumn{2}{p{10.0cm}}{ \LARGE{{ Milestones already met}}}\\
\hline \hline 
\multicolumn{2}{p{1.0cm}}{ \Large{{}}}\\
	02/05/2017 & \textbf{Final Protocol specification}\\
	&It has been decided the CPU transaction, the interrupt handling protocol,and the format of the 16 bits control word that the CPU writes at address \textit{0} for the IP manger. \\
	
	\multicolumn{2}{p{1.0cm}}{ \Large{{ }}}\\	
	12/05/2017 & \textbf{Enabling the right IP}	 \\
	&Given the physical address of a given IP-core, the IP-core manager has to enable the communication between the interested IP-core and the CPU while disabling the other cores.\\
		\multicolumn{2}{p{1.0cm}}{ \Large{{ }}}\\
		18/05/2017 & \textbf{Cross switching}\\
		& %Given $ N $ IP-cores, we have $ N $ $ data\_in $ array and $ N $ $ data\_out $ array. Each $ data\_in $ and $ data\_out $ array is connected to a single IP-core.
		A switch is needed when we have to propagate to the Data Buffer and to the CPU the right $ data\_in $, among the \textit{N}  $ data\_in $ of the \textit{N} IP-cores.
		The same thing is required in the other sense, when we have to send the data from the CPU to the selected IP-core, i.e. not to all of them.\\
		\multicolumn{2}{p{1.0cm}}{ \Large{{ }}}\\	
	20/05/2017 & \textbf{Dual-port Buffer 64x16}\\
	& We have written 2 version of the dual-port Buffer 64x16.\\
	& The first one is written with a behavioral architecture, while the second one is with a structural architecture.
	\\& The structural architecture is more complex because it requires to have all the subcomponent instantiated and connected in the right port between them.\\
	\multicolumn{2}{p{1.0cm}}{ \Large{{ }}}\\
		25/05/2017 & \textbf{Interrupt Handling}\\
		& The IP-core manager has to manage the case when a single or multiple core raise an interrupt request.\\
		& We will give priority to the core with the most priority level.
		\multicolumn{2}{p{1.0cm}}{ \Large{{ }}}\\
			\multicolumn{2}{p{1.0cm}}{ \Large{{ }}}
				\multicolumn{2}{p{1.0cm}}{ \Large{{ }}}\\
%\end{tabular}
%\end{table}	
%\begin{table}[h]
%	\begin{tabular}{p{1.9cm}|  p{15.5cm}}	
	\multicolumn{2}{p{10.0cm}}{ \LARGE{{Future developments and the planning}}}\\
	\hline \hline 
	\multicolumn{2}{p{1.0cm}}{ \Large{{}}}\\
	28/05/2017 & \textbf{Basic IP-Core}\\
	& Built a basic IP-core with the required interfaces.\\
	& This component will be used to test the correct behaviour of the IP-core manager\\
	\multicolumn{2}{p{1.0cm}}{ \Large{{ }}}\\		
	05/06/2017 & \textbf{Assemble all the components and Test}\\
	& Connect all the entities and the components.\\
	& Simulate a scenario for the IP-core manager and check the correct behaviour of everything.
	\\ \multicolumn{2}{p{1.0cm}}{ \Large{{ }}}\\	
		09/06/2017 & \textbf{Analyze the possibility to deploy on Secube board}\\
\end{tabular}
\end{table}	